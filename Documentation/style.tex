\usepackage[utf8]{inputenc} % Skal passe til editorens indstillinger
\usepackage[T1]{fontenc} % fonte (output)
\usepackage{lmodern} % vektor fonte
\usepackage{graphicx} % indsættelse af billeder
\graphicspath{{billeder/}} % stivej til bibliotek med figurer
\usepackage{mathtools} % matematik - understøtter muligheden for at bruge \eqref{}
\usepackage[draft,danish]{fixme}
\usepackage{sistyle}
\usepackage{blindtext}
\usepackage{tabularx}
\usepackage{pdflscape}
\usepackage[protrusion=true,expansion=true]{microtype}
\usepackage[hang,small,bf]{caption}
\usepackage{fancyvrb}
\usepackage{courier}
\usepackage[numbers]{natbib}
 \setcounter{secnumdepth}{3} % Paragraph bliver nummeret ligesom section
 \setcounter{tocdepth}{3}  % De bliver tilføjet til indeksering
% Indsæt todonotes i margin
\usepackage{todonotes}
\usepackage{pdfpages}
\SIdecimalsign{,}
\usepackage{textcomp}
\usepackage{listings}
\usepackage{color}
\usepackage{enumerate}
\usepackage{wrapfig}
\definecolor{listinggray}{gray}{0.9}
\definecolor{lbcolor}{rgb}{0.9,0.9,0.9}
\renewcommand\lstlistingname{Kodeeksempel}
\renewcommand\lstlistlistingname{Kodeeksempel}
%TEST
\newcommand\javakode{\renewcommand\lstlistingname{Kodeeksempel}}
\newcommand\pseudokode{\renewcommand\lstlistingname{Kodeeksempel}}
\newcommand\lstkode{\renewcommand\lstlistingname{Kodeeksempel}}

%CiteCMD
\def\citeAuthorNumber#1{[\citeauthor{#1}, \citep{#1}, \citeyear{#1}]}

% Custom commands
\newcommand{\sizeForIpadScreenShots}{0.2}
\newcommand{\newLine}{\newline\newline}


\lstset{
        basicstyle=\footnotesize\ttfamily, % Standardschrift
        numbers=left,               % Ort der Zeilennummern
        numberstyle=\color{blue!20!black!30!green}\tiny\ttfamily,          % Stil der Zeilennummern
        %stepnumber=2,               % Abstand zwischen den Zeilennummern
        numbersep=5pt,              % Abstand der Nummern zum Text
        tabsize=2,                  % Groesse von Tabs
        extendedchars=true,         %
        breaklines=true,            % Zeilen werden Umgebrochen
        keywordstyle=\color{red},
   		frame=b,         
%        keywordstyle=[1]\textbf,    % Stil der Keywords
%        keywordstyle=[2]\textbf,    %
%        keywordstyle=[3]\textbf,    %
%        keywordstyle=[4]\textbf,   \sqrt{\sqrt{}} %
        stringstyle=\color{blue}\ttfamily, % Farbe der String
        showspaces=false,           % Leerzeichen anzeigen ?
        showtabs=false,             % Tabs anzeigen ?
        xleftmargin=17pt,
        framexleftmargin=17pt,
        framexrightmargin=5pt,
        framexbottommargin=4pt,
        %backgroundcolor=\color{lightgray},
        showstringspaces=false,      % Leerzeichen in Strings anzeigen ?  
		inputencoding=utf8,
		  literate={å}{{\aa}}1
		                {æ}{{\ae}}1
		                 {ø}{{\o}}1      
}

\lstloadlanguages{% Check Dokumentation for further languages ...
        %[Visual]Basic
        %Pascal
        C,
        %C++
        %XML
        %HTML
        Java
}
\DeclareCaptionFont{white}{\color{white}}
\DeclareCaptionFormat{listing}{\colorbox[cmyk]{0.43, 0.35, 0.35,0.01}{\parbox{\textwidth}{\hspace{15pt}#1#2#3}}}
\captionsetup[lstlisting]{format=listing,labelfont=white,textfont=white, singlelinecheck=false, margin=0pt, font={bf,footnotesize}}

\usepackage{subfigure}
\usepackage{subfloat}
%\usepackage{geometry}
%\geometry{left=3.3cm,top=3cm,right=2.8cm,bottom=3.5cm}
\usepackage{booktabs,dcolumn,array}
\usepackage{multirow}
\renewcommand\multirowsetup{\centering}
\usepackage{cellspace}
	\addtolength\cellspacetoplimit{4pt}
	\addtolength\cellspacebottomlimit{4pt}
%\renewcommand\baselinestretch{1.6}

%\settocdepth{subsection}
%\setsecnumdepth{subsection}

\usepackage[plainpages=false,pdfpagelabels,pageanchor=false]{hyperref} % aktive links
%\hypersetup{%
\usepackage{memhfixc}% rettelser til hyperref

% -- Vis equations og figurere med chapter nummer først for bedre overskuelighed.
\numberwithin{equation}{chapter}
\numberwithin{figure}{chapter}

\usepackage{fancyhdr}


%Opsætning af marginer
\addtolength{\hoffset}{-0.3cm}
\addtolength{\textwidth}{2cm}
\addtolength{\voffset}{-1.8cm}
\addtolength{\textheight}{4.8cm}
\topmargin = 0cm
\setlength{\parindent}{0pt} % Sørger for at der ikke sker nogen indrykning ved nyt afsnit.

\pagestyle{fancyplain}

%Opsætninger af Header og Footer
\renewcommand{\chaptermark}[1]{\markboth{\thechapter.\ #1}{}}
\fancyhf{}
\lhead{\bfseries \leftmark}
\rhead{\bfseries \thepage}
\renewcommand{\headrulewidth}{0.5pt}
\renewcommand{\footrulewidth}{0.5pt}
\renewcommand{\headwidth}{15.9cm}
\addtolength{\headheight}{0.5pt}

\fancypagestyle{plain}{\renewcommand{\headrulewidth}{0.5pt}}
% Alle forsider er plain, og kan sættes op her.
\fancypagestyle{empty}{\renewcommand{\headrulewidth}{0pt} \renewcommand{\footrulewidth}{0pt} \lhead{} \rhead{}}

\fancypagestyle{lines}{\renewcommand{\headrulewidth}{0.5pt} \lhead{} \rhead{}}
% Siderne med Forord og Indholdsfortegnelse skal ikke have titel og nummerering, dette er opsat her.


%OpsÔøΩtning af Chapters
\makeatletter
\def\thickhrulefill{\leavevmode \leaders \hrule height 1ex \hfill \kern \z@}
\def\@makechapterhead#1{\vspace*{-10\p@}{\parindent \z@ {\raggedright \LARGE \bfseries \thechapter \;} \LARGE \bfseries #1}\vskip 10 \p@}
\def\@makeschapterhead#1{\vspace*{-10\p@}{\parindent \z@ {\raggedright \Huge \bfseries #1}\vskip 10 \p@}}

\setlength{\headheight}{14pt}

\usepackage[numbib,nottoc]{tocbibind}

